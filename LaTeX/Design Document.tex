\documentclass{article}
\usepackage[utf8]{inputenc}
\usepackage[margin=1in]{geometry}
\usepackage{array}
\usepackage{multirow}
\usepackage{makecell}
\usepackage{listings}

\title{Schwap CPU Design Documentation}
\author{Charlie Fenoglio, Alexander Hirschfeld, Andrew McKee, and Wesley Van Pelt}
\date{Winter 2015/2016}

\begin{document}
\maketitle
\section{Registers}
	There are a total of 76 16-bit registers; 12 are fixed and 64 (spilt into 16 groups of 4) "schwapable" registers.  Some registers have alias names, see Section 5.3.1 for a list.
	\subsection{Register Names and Descriptions}
		\begin{center}
			\begin{tabular}{| c | c | c | c |}
				\hline
				    Name        & Number  & Description            & Saved Across Call? \\ \hline
				    \$z0        & 0       & The Value 0$^\dagger$  & -   \\ \hline
				    \$a0        & 1       & Assembler Temporary 0  & No  \\ \hline
				    \$a1        & 2       & Assembler Temporary 1  & No  \\ \hline
				    \$pc        & 3       & Program Counter$^\ddagger$& Yes \\ \hline
				    \$sp        & 4       & Stack Pointer          & Yes \\ \hline
				    \$ra        & 5       & Return Address         & Yes \\ \hline
				    \$s0 - \$s1 & 6 - 7   & User Saved Temporaries & Yes \\ \hline
				    \$t0 - \$t3 & 8 - 11  & User Temporaries       & No  \\ \hline
				    \$h0 - \$h3 & 12 - 15 & Schwap                 & -   \\
				\hline
			\end{tabular} \\
			$^\dagger$See Section 5.1.1-1 for details\\
			$^\ddagger$See Section 5.1.1-2 for details
		\end{center}
	\subsection{Schwap Registers}
		The "schwap" registers are registers that appear to be swapped using a command.  There is no data movement when schwapping, it only changes which registers the \$h0 - \$h3 refer to.  There are 8 groups the user can use for general purpose and 8 reserved groups.
		\subsubsection{Schwap Group Numbers, Descriptions, and Uses}
			\begin{center}
				\begin{tabular}{| c | c | c |}
					\hline
				    	Group Number & Uses                      & Saved Across Call? \\ \hline
					    0 - 3        & User Temporaries          & No \\ \hline
					    4 - 7        & User Saved Temporaries    & Yes\\ \hline
					    8            & Arguments 0 - 3           & No \\ \hline
					    9            & Return Values 0 - 3       & No \\ \hline
					    10 - 14      & Reserved For Future Use$^\dagger$ & -  \\ \hline
					    15           & Go to the last used group & -  \\
					\hline
				\end{tabular} \\
				$^\dagger$See Section 5.1.2-1 for details
			\end{center}
\newpage
\section{Instructions}
	All instructions are 16-bits.  The destination register is also used as a source unless otherwise noted.  All offsets are bit shifted left by 1 since all instructions are 2 bytes long.
	\subsection{Instruction Types and Bit Layouts}
		Instructions can be manually translated by putting the bits for each of the components of the instructions in the places listed by the diagrams for each type.  The OP codes, function codes, and types can by found on the "Core Instructions Summary" (2.2.1) table.  The destination and source are register numbers, which can be found under the "Register Names and Descriptions" (1.1) table.  Schwap group numbers can be found under the "Schwap Group Numbers, Descriptions, and Uses" (1.2.1) table.  The active schwap group is not preserved over a function call.  See Section 5.2.1 for notes on the types and layouts.
		\subsubsection{A-Type}
			\begin{center}
				\begin{tabular}{l r l r l r l r}
					\hline
					\multicolumn{2}{| p{2cm} |}{OP Code} & \multicolumn{2}{p{2cm}}{Destination} & \multicolumn{2}{| p{2cm} |}{Source} & \multicolumn{2}{p{2cm} |}{Func. Code} \\ \hline
					15 & 12 & 11 & 8 & 7 & 4 & 3 & 0
				\end{tabular}
				\begin{tabular}{l r l r l r l r}
					\hline
					\multicolumn{2}{| p{2cm} }{Immediate} & \multicolumn{2}{p{2cm}}{ } & \multicolumn{2}{ p{2cm} }{ } & \multicolumn{2}{p{2cm} |}{ } \\ \hline
					15 & & & & & & & 0
				\end{tabular}
			\end{center}
			Used for all ALU operations.  It consists of a 4-bit OP code, 4-bit destination, 4-bit source, and a 4-bit function code.  If the instruction has an immediate, it is inserted as the next instruction.
		\subsubsection{B-Type}
			\begin{center}
				\begin{tabular}{l r l r l r l r}
					\hline
					\multicolumn{2}{| p{2cm} |}{OP Code} & \multicolumn{2}{p{2cm}}{R0} & \multicolumn{2}{| p{2cm} |}{R1} & \multicolumn{2}{p{2cm} |}{Offset} \\ \hline
					15 & 12 & 11 & 8 & 7 & 4 & 3 & 0
				\end{tabular}
			\end{center}
			If it is being used for branching it consists of a 4-bit OP code, 4-bit 1st source (R0), 4-bit 2nd source (R1), and a 4-bit (unsigned) offset. If it is being used for reading and writing to memory, it consists of a 4-bit OP code, 4-bit destination (R0 not used as a source), 4-bit source (R1), and a 4-bit (unsigned) offset.
		\subsubsection{H-Type}
			\begin{center}
				\begin{tabular}{l r l r l r l r}
					\hline
					\multicolumn{2}{| p{2cm} |}{OP Code} & \multicolumn{2}{p{2cm}}{ } & \multicolumn{2}{p{2cm}}{ } & \multicolumn{2}{| p{2cm} |}{Group} \\ \hline
					15 & 12 & & & & & 3 & 0
				\end{tabular}
			\end{center}
			Used for schwapping and sudo.  It consists of a 4-bit OP code, 8 unused bits, and a 4-bit schwap group number or sudo use case.
		\subsubsection{J-Type}
			Used for jumping.  It consists of a 4-bit OP code, 4-bit source, and an 8-bit (signed) offset.
			\begin{center}
				\begin{tabular}{l r l r l r l r}
					\hline
					\multicolumn{2}{| p{2cm} |}{OP Code} & \multicolumn{2}{p{2cm}}{Source} & \multicolumn{2}{| p{2cm} }{Offset} & \multicolumn{2}{p{2cm} |}{ } \\ \hline
					15 & 12 & 11 & 8 & 7 & & & 0
				\end{tabular}
			\end{center}
	\subsection{Core Instructions}
		Some instructions have alias names, see Section 5.3.2 for a list.
		\subsubsection{Core Instructions Summary}
		\begin{center} \begin{tabular}{| c | c | c | c | c | c |} \hline
			\thead{OP \\ Code} & \thead{Function \\ Code} & Name & Type & Description \\ \hline
			\multirow{15}{*}{\thead{0x0/1$^\dagger$}} & 0x0 & and & A & \thead{Bitwise ands 2 values}\\ \cline{2-5}
			     & 0x1 & orr  & A & \thead{Bitwise ors 2 values}\\ \cline{2-5}
			     & 0x2 & xor  & A & \thead{Bitwise xors 2 values}\\ \cline{2-5}
			     & 0x3 & not  & A & \thead{Bitwise nots 2 values}\\ \cline{2-5}
			     & 0x4 & tsc  & A & \thead{Converts a number to 2's compliment}\\ \cline{2-5}
			     & 0x5 & slt  & A & \thead{Set less than}\\ \cline{2-5}
			     & 0x6 & sll  & A & \thead{Left logical bit shift}\\ \cline{2-5}
			     & 0x7 & srl  & A & \thead{Right logical bit shift}\\ \cline{2-5}
			     & 0x8 & sra  & A & \thead{Right arithmetic bit shift}\\ \cline{2-5}
			     & 0x9 & add  & A & \thead{Adds 2 values}\\ \cline{2-5}
			     & 0xA & sub  & A & \thead{Subtracts 2 values}\\ \cline{2-5}
			     & 0xF & cpy  & A & \thead{Copies the value in one register to another}\\ \hline
			 0x2 & -   & beq  & B & \thead{Branches if the 2 values are equal}\\ \hline
			 0x3 & -   & bne  & B & \thead{Branches if the 2 values are not equal}\\ \hline
			 0x4 & -   & bgt  & B & \thead{Branches if value0 $>$ value1}\\ \hline
			 0x5 & -   & blt  & B & \thead{Branches if value0 $<$ value1}\\ \hline
			 0x6 & -   & jr   & J & \thead{Jumps to the value}\\ \hline
			 0x7 & -   & r    & B & \thead{Reads the value in memory into a register}\\ \hline
			 0x8 & -   & w    & B & \thead{Writes the value in a register into memory}\\ \hline
			 0xE & -   & rsh  & H & \thead{Changes the schwap group}\\ \hline
			 0xF & -   & sudo & H & \thead{Sames as "syscall" in MIPS} \\ \hline
		\end{tabular} \\
		\thead{$^\dagger$0x0 is used for instructions which do not use an immediate, 0x1 is used if the instruction does use an immediate}
		 \end{center}
		\subsubsection{Core Instructions and Details}
			\begin{itemize}
				\item[and:] A standard bitwise "and", it can use any of the syntaxes listed below.\\
					\begin{tabular}{| l | c | c |} \hline
						Syntax & Meaning & Description \\ \hline
						and [dest] [src]             & \thead{dest $=$ dest $\&$ src} & \thead{Bitwise ands the values in registers [dest] and [src]}\\ \hline
						and [dest] [src] [immediate] & \thead{src $=$ immediate \\ dest $=$ dest $\&$ src} & \thead{Loads the immediate into the register [src] and then \\ bitwise ands the values in registers [dest] and [src]}\\ \hline
						and [dest] [immediate]       & \thead{dest $=$ dest $\&$ immediate} & \thead{Bitwise ands the immediate and the value in \\ the register [dest]}\\ \hline
					\end{tabular}
				\item[orr:] A standard bitwise "or", it can use any of the syntaxes listed below.\\
					\begin{tabular}{| l | c | c |} \hline
						Syntax & Meaning & Description \\ \hline
						orr [dest] [src]             & \thead{dest $=$ dest $|$ src} & \thead{Bitwise ors the values in registers [dest] and [src]}\\ \hline
						orr [dest] [src] [immediate] & \thead{src $=$ immediate \\ dest $=$ dest $|$ src} & \thead{Loads the immediate into the register [src] and then \\ bitwise ors the values in registers [dest] and [src]}\\ \hline
						orr [dest] [immediate]       & \thead{dest $=$ dest $|$ immediate} & \thead{Bitwise ors the immediate and the value in \\ the register [dest]}\\ \hline
					\end{tabular}
				\item[xor:] A standard bitwise "xor", it can use any of the syntaxes listed below.\\
					\begin{tabular}{| l | c | c |} \hline
						Syntax & Meaning & Description \\ \hline
						xor [dest] [src]             & \thead{dest $=$ dest $\string^$ src} & \thead{Bitwise xors the values in registers [dest] and [src]}\\ \hline
						xor [dest] [src] [immediate] & \thead{src $=$ immediate \\ dest $=$ dest $\string^$ src} & \thead{Loads the immediate into the register [src] and then \\ bitwise xors the values in registers [dest] and [src]}\\ \hline
						xor [dest] [immediate]       & \thead{dest $=$ dest $\string^$ immediate} & \thead{Bitwise xors the immediate and the value in \\ the register [dest]}\\ \hline
					\end{tabular}
				\item[not:] A standard bitwise "not", it can use any of the syntaxes listed below.\\
					\begin{tabular}{| l | c | c |} \hline
						Syntax & Meaning & Description \\ \hline
						not [dest]                   & \thead{dest $=$ $\sim$dest} & \thead{Bitwise nots the value in the register [dest]}\\ \hline
						not [dest] [src] [immediate] & \thead{src $=$ immediate \\ dest $=$ $\sim$src} & \thead{Loads the immediate into the register [src] and then \\ bitwise nots the value in register [src] into [dest]}\\ \hline
						not [dest] [immediate]       & \thead{dest $=$ $\sim$immediate} & \thead{Bitwise nots the immediate into the register [dest]}\\ \hline
					\end{tabular}
				\item[tsc:] Converts an integer to/from 2's compliment, it can use any of the syntaxes listed below.\\
					\begin{tabular}{| l | c | c |} \hline
						Syntax & Meaning & Description \\ \hline
						tsc [dest]                   & \thead{dest $=$ $\sim$dest + 1} & \thead{Converts the value in the register [dest] to 2's compliment}\\ \hline
						tsc [dest] [src] [immediate] & \thead{src $=$ immediate \\ dest $=$ $\sim$src + 1} & \thead{Loads the immediate into the register [src] and then \\ converts the value in register [src] to 2's compliment and \\ stores into [dest]}\\ \hline
						tsc [dest] [immediate]       & \thead{dest $=$ $\sim$immediate + 1} & \thead{Converts the immediate to 2's compliment and stores \\ into the register [dest]}\\ \hline
					\end{tabular}
				\item[slt:] A standard ALU set less than, it can use any of the syntaxes listed below.\\
					\begin{tabular}{| l | c | c |} \hline
						Syntax & Meaning & Description \\ \hline
						slt [dest] [src]             & \thead{dest $=$ (dest $<$ src) ? 1 : 0} & \thead{If [dest] $<$ [src], then [dest] gets set to 1 \\ If [dest] $\geq$ [src], then [dest] gets set to 0}\\ \hline
						slt [dest] [src] [immediate] & \thead{src $=$ immediate \\ dest $=$ (dest $<$ src) ? 1 : 0} & \thead{Loads the immediate into the register [src] then \\ If [dest] $<$ [src], then [dest] gets set to 1 \\ If [dest] $\geq$ [src], then [dest] gets set to 0}\\ \hline
						slt [dest] [immediate]       & \thead{dest $=$ (dest $<$ immediate) ? 1 : 0} & \thead{If [dest] $<$ [immediate], then [dest] gets set to 1 \\ If [dest] $\geq$ [immediate], then [dest] gets set to 0}\\ \hline
					\end{tabular}
				\item[sll:] A standard shift left logical, it can use any of the syntaxes listed below.\\
					\begin{tabular}{| l | c | c |} \hline
						Syntax & Meaning & Description \\ \hline
						sll [dest] [shift]             & \thead{dest $=$ dest $<<$ shift} & \thead{Shifts the value in the register [dest] left \\ logically by [shift]}\\ \hline
						sll [dest] [shift] [immediate] & \thead{shift $=$ immediate \\ dest $<<$ shift} & \thead{Loads the immediate into the register [src] \\ and then shifts the value in the register [dest] \\ left logically by [shift]}\\ \hline
						sll [dest] [immediate]         & \thead{dest $=$ dest $<<$ immediate} & \thead{Shifts the value in the register [dest] \\ logically by [immediate]}\\ \hline
					\end{tabular}
				\item[srl:] A standard shift right logical, it can use any of the syntaxes listed below.\\
					\begin{tabular}{| l | c | c |} \hline
						Syntax & Meaning & Description \\ \hline
						srl [dest] [shift]             & \thead{dest $=$ dest $>>$ shift} & \thead{Shifts the value in the register [dest] right \\ logically by [shift]}\\ \hline
						srl [dest] [shift] [immediate] & \thead{shift $=$ immediate \\ dest $>>$ shift} & \thead{Loads the immediate into the register [src] \\ and then shifts the value in the register [dest] \\ right logically by [shift]}\\ \hline
						srl [dest] [immediate]         & \thead{dest $=$ dest $>>$ immediate} & \thead{Shifts the value in the register [dest] \\ right logically by [immediate]}\\ \hline
					\end{tabular}
				\item[sra:] A standard shift right arithmetic, it can use any of the syntaxes listed below.\\
					\begin{tabular}{| l | c | c |} \hline
						Syntax & Meaning & Description \\ \hline
						srl [dest] [shift]             & \thead{dest $=$ dest $>>>$ shift} & \thead{Shifts the value in the register [dest] right \\ arithmetically by [shift]}\\ \hline
						srl [dest] [shift] [immediate] & \thead{shift $=$ immediate \\ dest $>>>$ shift} & \thead{Loads the immediate into the register [src] \\ and then shifts the value in the register [dest] \\ right arithmetically by [shift]}\\ \hline
						srl [dest] [immediate]         & \thead{dest $=$ dest $>>>$ immediate} & \thead{Shifts the value in the register [dest] \\ right arithmetically by [immediate]}\\ \hline
					\end{tabular}
				\item[add:] A standard ALU add, it can use any of the syntaxes listed below.\\
					\begin{tabular}{| l | c | c |} \hline
						Syntax & Meaning & Description \\ \hline
						add [dest] [src]             & \thead{dest $=$ dest + src} & \thead{Adds the values in registers [dest] and [src]}\\ \hline
						add [dest] [src] [immediate] & \thead{src $=$ immediate \\ dest $=$ dest + src} & \thead{Loads the immediate into the register [src] and then \\ adds the values in registers [dest] and [src]}\\ \hline
						add [dest] [immediate]       & \thead{dest $=$ dest + immediate} & \thead{Adds the immediate and the value in \\ the register [dest]}\\ \hline
					\end{tabular}
				\item[sub:] A standard ALU subtract, it can use any of the syntaxes listed below.\\
					\begin{tabular}{| l | c | c |} \hline
						Syntax & Meaning & Description \\ \hline
						sub [dest] [src]             & \thead{dest $=$ dest $-$ src} & \thead{Subtracts the values in registers [dest] and [src]}\\ \hline
						sub [dest] [src] [immediate] & \thead{src $=$ immediate \\ dest $=$ dest $-$ src} & \thead{Loads the immediate into the register [src] and then \\ subtracts the values in registers [dest] and [src]}\\ \hline
						sub [dest] [immediate]       & \thead{dest $=$ dest $-$ immediate} & \thead{Subtracts the immediate and the value in \\ the register [dest]}\\ \hline
					\end{tabular}
				\item[cpy:] Copies the value from one register to another, or loads an immediate.  It can use any of the syntaxes listed below.\\
					\begin{tabular}{| l | c | c |} \hline
						Syntax & Meaning & Description \\ \hline
						cpy [dest] [src]             & \thead{dest $=$ src} & \thead{Copies the value the in register [src] into [dest]}\\ \hline
						cpy [dest] [immediate]       & \thead{dest $=$ immediate} & \thead{Loads the immediate into the register [dest]}\\ \hline
					\end{tabular}
				\item[beq:] Branches if the 2 values are equal. PC relative, if branching up or down more than 16 instructions, this will be a pseudo instruction.  It uses the syntax listed below.\\
					\begin{tabular}{| l | c | c |} \hline
						Syntax & Meaning & Description \\ \hline
						beq [src0] [src1] label      & \thead{if(src0 == src1) goto label} & \thead{If the values in the registers [src0] and [src1] \\ are equal, branch to label}\\ \hline
					\end{tabular}
				\item[bne:] Branches if the 2 values are not equal. PC relative, if branching up or down more than 16 instructions, this will be a pseudo instruction.  It uses the syntax listed below.\\
					\begin{tabular}{| l | c | c |} \hline
						Syntax & Meaning & Description \\ \hline
						bne [src0] [src1] label      & \thead{if(src0 != src1) goto label} & \thead{If the values in the registers [src0] and [src1] \\ are not equal, branch to label}\\ \hline
					\end{tabular}
				\item[bgt:] Branches if the value0 $>$ value1. PC relative, if branching up or down more than 16 instructions, this will be a pseudo instruction.  It uses the syntax listed below.\\
					\begin{tabular}{| l | c | c |} \hline
						Syntax & Meaning & Description \\ \hline
						bgt [src0] [src1] label      & \thead{if(src0 $>$ src1) goto label} & \thead{If [src0] $>$ [src1], branch to label}\\ \hline
					\end{tabular}
				\item[blt:] Branches if the value0 $<$ value1. PC relative, if branching up or down more than 16 instructions, this will be a pseudo instruction.  It uses the syntax listed below.\\
					\begin{tabular}{| l | c | c |} \hline
						Syntax & Meaning & Description \\ \hline
						blt [src0] [src1] label      & \thead{if(src0 $<$ src1) goto label} & \thead{If [src0] $<$ [src1], branch to label}\\ \hline
					\end{tabular}
				\item[jr:] Jumps to an instruction, the offset is signed.  It can use any of the syntaxes listed below.\\
					\begin{tabular}{| l | c | c |} \hline
						Syntax & Meaning & Description \\ \hline
						jr [offset]([dest]) & \thead{pc $=$ dest + offset} & \thead{Jumps to the instruction at the address in [dest] + [offset]}\\ \hline
					\end{tabular}
				\item[r:] Reads a value in memory, it uses the syntax listed below.\\
					\begin{tabular}{| l | c | c |} \hline
						Syntax & Meaning & Description \\ \hline
						r [dest] [offset]([src])     & \thead{dest $=$ Mem[src + offset]} & \thead{Reads the data in the address of [src] + [offset] \\ in memory into [dest]}\\ \hline
					\end{tabular}
				\item[w:] Writes a value in memory, it uses the syntax listed below.\\
					\begin{tabular}{| l | c | c |} \hline
						Syntax & Meaning & Description \\ \hline
						w [offset]([dest]) [src]     & \thead{Mem[dest + offset] $=$ src} & \thead{Writes the data in the address of [dest] + [offset] \\ in memory from [src]}\\ \hline
					\end{tabular}
				\item[rsh:] Sets the schwap group, it uses the syntax listed below.\\
					\begin{tabular}{| l | c | c |} \hline
						Syntax & Meaning & Description \\ \hline
						rsh [group]     & \thead{SchwapGroup $=$ group} & \thead{Changes the schwap group number to [group], \\ these numbers can be found in the table in 1.2.1}\\ \hline
					\end{tabular}
				\item[sudo:] Does what "syscall" in MIPS does, just type "sudo" for the instruction.
			\end{itemize}
	\subsection{Pseudo Instructions}
		There are two types of pseudo instructions.  One are instructions which are always pseudo instructions, the other are sometimes pseudo depending on the conditions.  Some instructions have alias names, see Section 5.3.2 for a list.
		\subsubsection{Always Pseudo Instructions}
			\begin{center} \begin{tabular}{| c | c | c | c |} \hline
				Name & Syntax    & Actual Code & Description \\ \hline
				j    & j label   & \thead{cpy \$a0 [label pc] \\ jr 0(\$a0)} & \thead{Jumps to the instruction at label}\\ \hline
				jal  & jal label & \thead{cpy \$ra \$pc \\ j [label]} & \thead{Stores the return address and then jumps to the label} \\ \hline
				bge  & bge [src0] [src1] label & \thead{cpy \$a0 [src0] \\ slt \$a0 [src1] \\ beq \$a0 \$z0 label} & \thead{If [src0] $\geq$ [src1], branch to label} \\ \hline
				ble  & ble [src0] [src1] label & \thead{cpy \$a0 [src1] \\ slt \$a0 [src0] \\ beq \$a0 \$z0 label} & \thead{If [src0] $\leq$ [src1], branch to label} \\ \hline
			\end{tabular} \end{center}
		\subsubsection{Conditional Pseudo Instructions}
			\begin{center} \begin{tabular}{| c | c | c | c |} \hline
				Name & Syntax                    & Actual Code & Condition \\ \hline
				beq  & beq [src0] [src1] label & \thead{bnq [src0] [src1] Next \\ j label \\ Next:} & \thead{Branching up or branching down \\ more than 16 instructions}\\ \hline
				bne  & bne [src0] [src1] label & \thead{beq [src0] [src1] Next \\ j label \\ Next:} & \thead{Branching up or branching down \\ more than 16 instructions}\\ \hline
				bgt  & bgt [src0] [src1] label & \thead{blt [src0] [src1] Next \\ j label \\ Next:} & \thead{Branching up or branching down \\ more than 16 instructions}\\ \hline
				blt  & blt [src0] [src1] label & \thead{bgt [src0] [src1] Next \\ j label \\ Next:} & \thead{Branching up or branching down \\ more than 16 instructions}\\ \hline
			\end{tabular} \end{center}
\section{Assembler and Coding Practices}
	
\section{Examples}
	\lstset{language=sh}
	\subsection{Basic Use Examples}
		\subsubsection{Loading an immediate into a register}
			\begin{lstlisting}[frame=single]
cpy $t0 32	# Loads 32 into t0
			\end{lstlisting}
		\subsubsection{Making a Procedure Call}
			\begin{lstlisting}[frame=single]
rsh 8		# Switch to arguments schwap
cpy $h0 $t0	# Put argument0 in
cpy $h1 $s1	# Put argument1 in
# Store any wanted temporaries somewhere
jal Call
rsh 9		# Switch to return values schwap
cpy $s0 $h0	# Copy the return values out
			\end{lstlisting}
		\subsubsection{Iteration and Conditionals}
			This is an example of which will iterate over 4 array elements in memory and add 32 to each of them.  It will stop repeating after the 4 elements using beq.
			\begin{lstlisting}[frame=single]
	# There is a base memory address for an array in memory at s0
	cpy $t0 8
	cpy $t1 $z0
loop:
	r   $t2 0($s0)
	add $t2 32
	w   0($s0) $t2
	add $t1 2
	beq $t0 $t1 loop
			\end{lstlisting}
	\subsection{relPrime and gcd Implementation}
		\subsubsection{Assembly}
			\begin{lstlisting}[frame=single]
RelPrime:
	rsh	8		#set schwap
	cpy	$s2 $ra		#save $ra
	cpy	$s0 $h0		#copy n out of schwap
	cpy	$s1 0x2		#load 2 to m
	rsh	8		#set schwap to args
While:
	cpy	$h0 $s0		#set a0 to n
	cpy	$h1 $s1		#set a1 to m
	jal	GCD		#call GCD
	rsh	9		#set schwap
	cpy	$t0 0x1		#load immediate 0x1 to t0
	bne	$h0 $t0 Done	#branch to done if r0 != 1
	add	$s1 0x1		#add 1 to m
	j	While		#jump to the start of the loop
Done:
	rsh	9		#load return registers
	cpy	$h0 $s1		#set r0 to m
	j	$s2 0		#return to the previous function
			\end{lstlisting}
			\begin{lstlisting}[frame=single]
GCD:
	rsh	8		#schwap to argument register
Base:
	bne	$h0 $z0 GMain	#a!=0 go to GMain
	cpy	$t0 $h1		#copy h1 to t0 for RSH
	rsh	9		#schwap to return registers
	cpy	$h0 $t0		#load t0 to r1
	j	$ra 0		#return
GMain:
	beq	$h1 $z0 Exit	#jump to exit if b is zero
	bgt	$h0 $h1 If	#jump to If if a>b
Else:
	sub	$h1 $h0		#else: b=b-a
	j	GMain		#loop
If:
	sub	$h0 $h1		#if: a=a-b
	j	GMain		#loop
Exit:
	cpy	$t0 $h0		#copy h0 to t0 for rsh schwap
	rsh	9		#make sure we're in the right spot
	cpy	$h0 $t0		#copy t0 to h0
	j	$ra		#return
			\end{lstlisting}
	\subsection{Machine Code}
		\begin{center}
			\begin{tabular}{| c | c || c | c |} \hline
				\multicolumn{2}{| p{1.5cm} ||}{RelPrime} & \multicolumn{2}{ p{1.5cm} |}{GCD} \\ \hline
				PC & Hex  & PC & Hex \\ \hline
				00 & 3009 & 42 & 3009 \\ \hline
				02 & 063F & 44 & 5C05 \\ \hline
				04 & 04CF & 46 & 085F \\ \hline
				06 & 150F & 48 & 300A \\ \hline
				08 & 0002 & 4A & 0A8F \\ \hline
				0A & 3009 & 4C & 2300 \\ \hline
				0C & 0C4F & 4E & 4B05 \\ \hline
				0E & 0D5F & 50 & 1C0F \\ \hline
				10 & 031F & 52 & 0072 \\ \hline
				12 & 300F & 54 & 2C00 \\ \hline
				14 & 1C0F & 56 & 301F \\ \hline
				16 & 0042 & 58 & 6CD7 \\ \hline
				18 & 2C00 & 5A & 0DC2 \\ \hline
				1A & 301F & 5C & 300F \\ \hline
				1C & 300A & 5E & 1C0F \\ \hline
				1E & 180F & 60 & 004E \\ \hline
				20 & 0001 & 62 & 2C00 \\ \hline
				22 & 4C85 & 64 & 301F \\ \hline
				24 & 1C0F & 66 & 0CD2 \\ \hline
				26 & 000A & 68 & 300F \\ \hline
				28 & 2C00 & 6A & 1C0F \\ \hline
				2A & 301F & 6C & 004E \\ \hline
				2C & 1500 & 6E & 2C00 \\ \hline
				2E & 0001 & 70 & 301F \\ \hline
				30 & 300F & 72 & 08CF \\ \hline
				34 & 1C0F & 74 & 300A \\ \hline
				36 & 000C & 76 & 0C8F \\ \hline
				38 & 2C00 & 78 & 2300 \\ \hline
				3A & 301F & & \\ \hline
				3C & 300A & & \\ \hline
				3E & 0C5F & & \\ \hline
				40 & 2600 & & \\ \hline
			\end{tabular}
		\end{center}
\newpage
\section{Notes}
	\subsection{Registers}
		\subsubsection{Non-Schwappable}
			\begin{enumerate}
				\item \$z0 is reset on the rising edge of each CPU cycle, so it can be used for cycle-temporary storage.
				\item The value in \$pc should always be what the current instruction address is +2.
			\end{enumerate}
		\subsubsection{Schwappable}
			\begin{enumerate}
				\item Possible uses for the reserved for future use groups:
					\begin{center} \begin{tabular}{| c | c | c |} \hline
						Group Number & ID    & Use \\ \hline
						\multirow{4}{*}{10}  & 0 & The constant 1 \\
						             & 1     & The constant -1 \\
						             & 2     & User set constant0 \\
						             & 3     & User set constant1 \\ \hline
						11           & 0 - 3 & I/O for devices 0 - 3 \\ \hline
						12           & 0 - 3 & Syscall values 0 - 3 \\ \hline
						13           & 0 - 3 & Kernel \\ \hline
						\multirow{4}{*}{14}  & 0 & Exception Cause \\
						             & 1     & Exception Status \\
						             & 2     & EPC \\
						             & 3     & Exception Temporary \\ \hline
					\end{tabular} \end{center} 
			\end{enumerate}
	\subsection{Instructions}
		\subsubsection{Types and Layouts}
			\begin{enumerate}
				\item More of the types could be combined, but they will run faster if they are not.
			\end{enumerate}
	\subsection{Alias names}
		\subsubsection{Registers}
			\begin{itemize}
				\item[\$z0:] \$0, \$00, \$zz, \$zero
			\end{itemize}
		\subsubsection{Instructions}
			\begin{itemize}
				\item[orr:] or
				\item[bne:] bnq
			\end{itemize}
\end{document}
