\documentclass{article}
\usepackage[utf8]{inputenc}
\usepackage[margin=1in]{geometry}
\usepackage{array}
\usepackage{multirow}

\title{Shwap CPU Design Documentation}
\author{fenogljc, hirschag, mckeeaj, and Wesley Van Pelt}
\date{Winter 2015/2016}

\begin{document}
\maketitle
\section{Registers}
	There are a total of 76 16-bit registers; 12 are fixed and 64 (spilt into 16 groups of 4) "shwapable" registers.
	
	\subsection{Register Names and Discriptions}
		\begin{center}
			\begin{tabular}{| c | c | c | c |}
				\hline
				    Name        & Number  & Description     & Saved Across Call? \\ \hline
				    \$0         & 0       & The Value 0     & -   \\ \hline
				    \$pc        & 1       & Program Counter & Yes \\ \hline
				    \$sp        & 2       & Stack Pointer   & Yes \\ \hline
				    \$ra        & 3       & Return Address  & Yes \\ \hline
				    \$s0 - \$s3 & 4 - 7   & Saved           & Yes \\ \hline
				    \$t0 - \$t3 & 8 - 11  & Temporaries     & No  \\ \hline
				    \$h0 - \$h3 & 12 - 15 & Shwap           & -   \\ 
				\hline
			\end{tabular}
		\end{center}
	\subsection{Shwap Registers}
		The "shwap" registers are registers that appear to be swapped using a command.  There is no data movement when shwapping, it only changes which registers the \$h0 - \$h3 refer to.  There are 8 groups the user can switch between and 8 reserved groups.
		\subsubsection{Shwap Group Numbers, Descriptions, and Uses}
			\begin{center}
				\begin{tabular}{| c | c | c | c |}
					\hline
				    	Group Number   & ID    & Uses                     & Saved Across Call? \\ \hline
					    0 - 7          & 0 - 3 & User Temporaries         & No \\ \hline
					    8              & 0 - 3 & I/O for devices 0 - 3    & -  \\ \hline
					    9              & 0 - 3 & Arguments 0 - 3          & No \\ \hline
					    10             & 0 - 3 & Return Values 0 - 3      & No \\ \hline
					    11             & 0 - 3 & System Call Values 0 - 3 & No \\ \hline
				    	12             & 0 - 3 & Kernel Reserved          & No \\ \hline
					    13             & 0 - 3 & Temporary Restore        & No \\ \hline 
					    \multirow{4}{*}{14} & 0 & Exception Cause         & No \\
					    			   & 1     & Exception Status         & No \\
								       & 2     & EPC                      & No \\
								       & 3     & Exception Temporary      & No \\ \hline
					    15             & 0 - 3 & Assembler Temporaries    & No \\
					\hline
				\end{tabular}
			\end{center}
\section{Instructions}
	\subsection{Instruction Types and Bit Layouts}
		[stuff]
	\subsection{Core Instructions}
		[stuff]
	\subsection{Sudo Instructions}
		[stuff]

\end{document}
